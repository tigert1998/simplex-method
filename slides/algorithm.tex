\section{Algorithm}

\subsection{Slack form}

\begin{frame}{Converting linear programs into slack form}

    \begin{gather*}
        \sum_{j=1}^{n}a_{ij}x_j \le b_i\\
        x_{n+i} = b_{i} - \sum_{j=1}^{n}a_{ij}x_j\\
        x_i\ge 0 \text{ for } i = 1,2,\cdots,n+m
    \end{gather*}

\end{frame}

\subsection{Definitions}

\begin{frame}{Definitions}

    \begin{itemize}
        \item We call the variables on the left-hand side of
              the equalities \textbf{basic variables} and those on the right-hand side
              \textbf{nonbasic variables}.
        \item We use $N$ to denote the set of indices of the nonbasic variables
              and $B$ to denote the set of indices of the basic variables.
              We always have that $|N|=n,|B|=m$, and $N\cup B=\{1, 2,\cdots, n+m\}$.
        \item \textbf{basic solution}: set all the (nonbasic) variables on the
              right-hand side to $0$ and then compute the values of the (basic) variables
              on the left-hand side.
    \end{itemize}

\end{frame}

\subsection{Simplex algorithm}

\begin{frame}{Simplex}

    In geometry, a simplex (plural: simplexes or simplices) is a
    generalization of the notion of a triangle or tetrahedron to arbitrary dimensions.
    Specifically, a $k$-simplex is a $k$-dimensional polytope which is
    the convex hull of its $k + 1$ vertices.

    \begin{figure}
        \includegraphics[width=0.7\textwidth]{assets/simplex.png}
        \caption{Simplex}
    \end{figure}

\end{frame}

\begin{frame}{Simplex algorithm $\Rightarrow$ Start with an example!}

    \begin{block}{Standard form}
        maximize $$3x_1 + x_2 + 2x_3$$
        subject to
        \begin{align*}
            x_1 + x_2 + 3x_3   & \le 30 \\
            2x_1 + 2x_2 + 5x_3 & \le 24 \\
            4x_1 + x_2 + 2x_3  & \le 36
        \end{align*}
        $$x_1, x_2, x_3\ge 0$$
    \end{block}

\end{frame}

\begin{frame}{Simplex algorithm $\Rightarrow$ Start with an example!}

    \begin{block}{Convert to slack form}
        maximize $$z = 3x_1 + x_2 + 2x_3$$
        subject to
        \begin{align*}
            x_4 & = 30 - x_1 - x_2 - 3x_3   \\
            x_5 & = 24 - 2x_1 - 2x_2 - 5x_3 \\
            x_6 & = 36 - 4x_1 - x_2 - 2x_3  \\
        \end{align*}
        $$x_1, x_2, x_3, x_4, x_5, x_6\ge 0$$
    \end{block}

    Our goal, in each iteration, is to reformulate the linear program so that the basic
    solution has a greater objective value.

\end{frame}

\begin{frame}{Simplex algorithm $\Rightarrow$ Start with an example!}

    \begin{block}{Pivot($x_e = x_1$, $x_l = x_6$)}
        \begin{align*}
            z   & = 27 + \frac{x_2}{4} + \frac{x_3}{2} - \frac{3x_6}{4}  \\
            x_1 & = 9 - \frac{x_2}{4} - \frac{x_3}{2} - \frac{x_6}{4}    \\
            x_4 & = 21 - \frac{3x_2}{4} - \frac{5x_3}{2} + \frac{x_6}{4} \\
            x_5 & = 6 - \frac{3x_2}{2} - 4x_3 + \frac{x_6}{2}            \\
        \end{align*}
        Here we have $B = \{1, 4, 5\}, N = \{2, 3, 6\}$
    \end{block}

    A \textbf{pivot} chooses a nonbasic variable $x_e$,
    called the \textbf{entering variable}, and a basic variable $x_l$,
    called the \textbf{leaving variable}, and exchanges their roles.

\end{frame}

\begin{frame}{Simplex algorithm $\Rightarrow$ Start with an example!}

    After performing Pivot($x_e = x_3$, $x_l = x_5$)
    and Pivot($x_e = x_2$, $x_l = x_3$), we get:
    \begin{align*}
        z   & = 28 - \frac{x_3}{6} - \frac{x_5}{6} - \frac{2x_6}{3} \\
        x_1 & = 8 + \frac{x_3}{6} + \frac{x_5}{6} - \frac{x_6}{3}   \\
        x_2 & = 4 - \frac{8x_3}{3} - \frac{2x_5}{3} + \frac{x_6}{3} \\
        x_4 & = 18 - \frac{x_3}{2} + \frac{x_5}{2}                  \\
    \end{align*}
    Here we have $B = \{1, 2, 4\}, N = \{3, 5, 6\}$\\
    $\because c_j < 0$ for $j\in N$
    $\therefore$ answer is $28$

\end{frame}

\begin{frame}{Issues}

    \begin{itemize}
        \item How do we choose the entering and leaving variables?
        \item How do we determine whether a linear program is unbounded?
        \item How do we determine whether a linear program is feasible?
        \item What do we do if the linear program is feasible,
              but the initial basic solution is infeasible?
    \end{itemize}

\end{frame}

\begin{frame}{$\exists b_i<0$}

    The basic solution is not feasible anymore!
    \begin{block}{auxiliary linear program}
        maximize $$-x_0$$
        subject to
        \begin{align*}
            \sum_{j=1}^n a_{ij}x_j-x_0 & \le b_i & \text{ for } i =1,2,\cdots, m \\
            x_j                        & \ge 0   & \text{ for } j = 0,1,\cdots,n
        \end{align*}
        Then $L$ is feasible if and only if the optimal objective value of $L_{aux}$ is 0.
        If feasible, $L_{aux}$ provides a feasible solution 
        with which we can initialize $L$.
    \end{block}

\end{frame}

\begin{frame}{Others}
    \begin{block}{Termination}
        \begin{itemize}
            \item at most $\binom{n + m}{m}$ iterations
            \item may not terminate!
            \item \textbf{Bland’s rule}
        \end{itemize}
    \end{block}

    \begin{block}{Why always the optimal solution?}
        \textbf{linear-programming duality}
    \end{block}
\end{frame}

\subsection{Analysis}

\begin{frame}{Analysis}

    \begin{itemize}
        \item Not a polynomial time algorithm. Minty constructed an example
              on which the simplex algorithm runs through $2^n-1$ iterations.
        \item The ellipsoid algorithm was the first polynomial-time
              algorithm for linear programming.
        \item To date, the ellipsoid algorithm does not appear to be
              competitive with the simplex algorithm in practice.
    \end{itemize}

\end{frame}
